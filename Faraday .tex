\documentclass[a4paper,12pt]{article}
\usepackage[utf8]{inputenc}
\usepackage{amsmath, amssymb}

\title{Lei de Faraday-Lenz}
\author{}
\date{}

\begin{document}

\maketitle

\section*{Lei de Faraday-Lenz}

\begin{itemize}
    \item É a base do funcionamento de um gerador elétrico, dínamo, alternador, etc.
    \item Suponha que você tenha uma espira de fio.
\end{itemize}

Se o campo magnético varia, e a espira tem uma área $A$, o fluxo do campo magnético por ela é:

\[
\Phi_B = B \cdot A
\]

Onde $B$ é a intensidade do campo na direção da área, e $A$ é a área.  
A unidade de $\Phi_B$ é o $T \cdot m^2$, que recebe o nome de \textbf{weber (Wb)}.

A Lei de Faraday diz que, se o fluxo varia, surge na espira uma tensão induzida, dada por:

\[
\varepsilon = - \frac{\Delta \Phi_B}{\Delta t}
\]

O sinal negativo indica que a tensão induzida se opõe à variação que a origina (Lei de Lenz).

\section*{Exemplo}

O fluxo do campo varia de $50\,\text{Wb}$ para $100\,\text{Wb}$ em $10\,\text{s}$.  
Qual é a tensão induzida?

\subsection*{Solução}

\[
\varepsilon = -\frac{\Delta \Phi}{\Delta t} = -\frac{100 - 50}{10}
\]

\[
\varepsilon = -\frac{50}{10} = -5 \,\text{V}
\]

\end{document}
