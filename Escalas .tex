\documentclass[a4paper,12pt]{article}
\usepackage[utf8]{inputenc}
\usepackage{amsmath, amssymb}

\title{Escalas Termométricas}
\author{}
\date{}

\begin{document}

\maketitle

\section*{Escalas Termométricas}

As três escalas de temperatura mais usadas são:
\begin{itemize}
    \item Escala Celsius
    \item Escala Fahrenheit
    \item Escala Kelvin
\end{itemize}

Originalmente as escalas foram baseadas em duas temperaturas fáceis de reproduzir, chamadas de \textbf{pontos fixos}.

\begin{itemize}
    \item 1º ponto fixo: ponto do gelo (é a temperatura em que o gelo, sob pressão ambiente, derrete).  
    Corresponde a $0^\circ$C ou $32^\circ$F.
    \item 2º ponto fixo: ponto do vapor (é a temperatura em que a água, sob pressão normal, ferve).  
    Corresponde a $100^\circ$C ou $212^\circ$F.
\end{itemize}

A escala Celsius divide o intervalo entre os pontos fixos em 100 partes (graus Celsius), enquanto a escala Fahrenheit divide em 180 partes (graus Fahrenheit).  

O Kelvin tem o zero em $-273{,}15^\circ$C, que é a menor temperatura existente, chamada \textbf{zero absoluto}:

\[
0\,K = -273{,}15^\circ C
\]

\section*{Conversão entre escalas}

Para converter entre as escalas, usamos:

\[
\frac{T_C}{5} = \frac{T_F - 32}{9}
\]

\[
T_K = T_C + 273{,}15
\]

Onde:
\begin{itemize}
    \item $T_C =$ Temperatura em Celsius
    \item $T_F =$ Temperatura em Fahrenheit
    \item $T_K =$ Temperatura em Kelvin
\end{itemize}

\end{document}
