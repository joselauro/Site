\documentclass[12pt]{article}
\usepackage{amsmath}
\usepackage{amssymb}
\usepackage{siunitx}

\begin{document}

\section*{Pressão}

Suponha que você tenha uma força $\vec{F}$ uniformemente distribuída sobre uma área $A$, e perpendicular a esta:

\[
p = \frac{F}{A}
\]

Chamamos de pressão $P$ (ou $p$) a razão entre o módulo de força e a área.

A unidade de pressão no SI é o newton por metro quadrado ($\si{N/m^2}$), que recebe o nome de pascal (Pa).

\subsection*{Outras unidades}
Unidades comuns são a libra por polegada quadrada (psi), a atmosfera, o milímetro de mercúrio, etc.

\[
1 \, \text{psi} = 6895 \, \text{Pa}
\]
\[
1 \, \text{atm} = 1{,}01 \times 10^5 \, \text{Pa}
\]
\[
1 \, \text{mmHg} = 133 \, \text{Pa}
\]

Obs.: $1 \, \text{atm} = 760 \, \text{mmHg}$

\subsection*{Exemplo}
Uma força de \SI{50}{N} atua perpendicularmente sobre uma área quadrada de \SI{20}{cm} $\times$ \SI{20}{cm}. Qual a pressão?

\[
A = 0{,}20 \times 0{,}20 = 0{,}09 \, \text{m}^2
\]

\[
p = \frac{F}{A} = \frac{50}{0{,}09} \approx 1250 \, \text{Pa}
\]

\end{document}
